\documentclass[11pt]{article}

\usepackage{amssymb,amsmath,exscale}
\usepackage[utf8]{inputenc}

% \usepackage[]{authblk}
\usepackage{palatino}
\usepackage{graphicx}
\usepackage{amssymb}
\usepackage{amsfonts}
\usepackage{amsbsy}
% \usepackage{psfrag}
%\usepackage{float}
\usepackage{color}
\usepackage[english]{babel}
%%\usepackage[latin1]{inputenc}
%\usepackage[applemac]{fontenc}
\usepackage{float}
\usepackage{url}
\usepackage{hyperref}
\usepackage{subfigure}
\usepackage{commands}
%\usepackage[french]{babel}
\author{}

\renewcommand{\thesection}{\arabic{section}}

\topmargin -1cm
\oddsidemargin 0cm
\evensidemargin 0cm
\textwidth 16.5cm
\textheight 23.5cm

\def\theequation{\thesection.\arabic{equation}}

\newtheorem{remark}{Remark}[section]
\def\begrem{\begin{remark}\rm}
\def\endrem{\null\hfill\blackbox\end{remark}}

\newtheorem{prop}{Proposition}[section]

\def\NN{{\rm I\hspace{-0.50ex}N} }
\def\RR{{\rm I\hspace{-0.50ex}R} }
\def\be{\begin{equation}}
\def\ee{\end{equation}}
\let\pa\partial
\def \intdb{\int\!\!\!\!\int}

\title{Datas}
\date{}
\newtheorem{theorem}{Theorem}
\newtheorem{lemma}{Lemma}

% \includeonly{MicroMacro}
% \usepackage{setspace}

%
\usepackage{nicefrac}
\definecolor{gris}{gray}{.99}
\usepackage{listings}
\lstset{ %
language=[90]Fortran,                % the language of the code
basicstyle=\footnotesize,       % the size of the fonts that are used for the code
numbers=none,                   % where to put the line-numbers
numberstyle=\footnotesize,      % the size of the fonts that are used for the line-numbers
stepnumber=2,                   % the step between two line-numbers. If it's 1, each line
                                % will be numbered
numbersep=5pt,                  % how far the line-numbers are from the code
backgroundcolor=\color{gris},  % choose the background color. You must add
showspaces=false,               % show spaces adding particular underscores
showstringspaces=false,         % underline spaces within strings
showtabs=false,                 % show tabs within strings adding particular underscores
frame=single,                   % adds a frame around the code
tabsize=2,                      % sets default tabsize to 2 spaces
captionpos=b,                   % sets the caption-position to bottom
breaklines=true,                % sets automatic line breaking
breakatwhitespace=false,        % sets if automatic breaks should only happen at whitespace
title=\lstname,                 % show the filename of files included with \lstinputlisting;
frameround={tttt},                                % also try caption instead of title
escapeinside={\%*}{*)},         % if you want to add a comment within your code
morekeywords={*,...}            % if you want to add more keywords to the set
}
\begin{document}
\maketitle
%\section{Datas/Values}
\begin{center}
\begin{tabular}{|c|p{0.5\textwidth}|l|}
\hline
Notation & Definition (according to \cite{lyubimov2007dynamics}) & Value \\
\hline
\hline
$U$ & Total aqueous humor & \\
\hline
$F_h$ & Fluid inflow in posterior chamber & mean daily value : $2.4 \mathrm{mm}^3/\mathrm{min}$ \\
&& datas : $2.1$ -- $4.0 \mathrm{mm}^3/min$\\
\hline
$F_e$ & net inflow via trabecular path & \\
 \hline
 $p$ & IOP & $14$ -- $30 \mathrm{mmHg}$\\
 \hline
 $R$ & output hydraulic resistance & $2.5$--$5 \mathrm{mmHgmin}/\mathrm{mm}^3$ \cite{lyubimov2007dynamics}\\
 && $4\mathrm{mmHgmin}/\mathrm{mm}^3$ (mean) \cite{bill1975blood}\\
 \hline
 $p_e$ & pressure in the episcleral veins & $4$ -- $8$ $\mathrm{mmHG} $\\
 \hline
 $p_a$ & pressure in the ciliary body capillaries & $30$ -- $35 \mathrm{mmHg}$ \cite{lyubimov2007dynamics}\\
 &we can find it by using the formula $p_a = 1/3 \mathrm{MAP} $ \textit{REF?} & HBP : $35.5 \mathrm{mmHg} $ -- NBP : $31.1\mathrm{mmHg}$  \\
 && LBP : $26.6\mathrm{mmHg}$\\
 \hline
 $L_p$ & permeability of the equivalent membrane & $0.3 \mathrm{mm^3}/\mathrm{minmmHg}$\\
 \hline
 $\Delta \Pi_p $& osmotic pressure diff. accross membrane (proteins) &$\approx 25 \mathrm{mmHg}$\\
 \hline
 $\Delta \Pi_s $&osmotic pressure diff. accross membrane (low molecular component)& $-400$ -- $-450 \mathrm{mmHg}$\\
 \hline
 $\sigma_p$ &reflection coeff. (proteins) & $1$ \\
 \hline
 $\sigma_s$ & reflection coeff. (low molecular components) & \textit{must range} $0.02$--$0.2$ \\
 \hline
$ C_{p_1}$& proteins concentration at membrane output (blood plasma) & $\approx 7 \mathrm{g}/\mathrm{dl}$ \\
\hline
$C_{p_2}$ & proteins concentration at membrane output (aqueous humor) &$\approx 0$ \\
\hline
$C_p$ & proteins concentration & $\equiv C_{p_1}$ \\
\hline
$\rho$ & universal gas constant $\times$  absolute temperature &$8.31 \times 300 \mathrm{J}/\mathrm{mol} $\\
&& $2493 \mathrm{N}.\mathrm{m}/\mathrm{mol}$\\
\hline
$C_1$& total molar concentration of low-molecular components (blood) & $306.64.10^{-3} \mu \mathrm{mol}/\mathrm{mm}^3$ \textbf{??}\cite{to2002mechanism}\\\\
\hline
$C_2$& total molar concentration of low-molecular components (intra-ocular fluid near ciliary body surface) & \\
\hline
$C_2^0$&Initial possible value of $C_2$&$325.02 \mu \mathrm{mol}/\mathrm{mL}$ \\
& & $ = 325.02.10^{-3}	 \mu \mathrm{mol}/\mathrm{mm^3}$
\cite{to2002mechanism}\\
\hline
$Q_s$ & Flux of these low-molecular components & \\
\hline
$J$ & Influx due to active transport & $\approx 0.04$--$0.18 \mu \mathrm{mol} / \mathrm{min}$\\
\hline
$\xi_s$ & average permeability of membrane for low-molecular species& \\
\hline
$\overline{C}$& $\displaystyle{\frac{C_1+C_2}{2}}$& \\
\hline

$V^\star$ & volume of intraocular fluid between the folds of the ciliary body & $20 \mathrm{mm}^3 $\\
\hline

\end{tabular}
\end{center}
\begin{center}
\begin{tabular}{|c|p{0.5\textwidth}|l|}
\hline
Notation & Definition (according to \cite{lyubimov2007dynamics}) & Value \\
\hline
\hline
$2l$& characteristic distances between the folds of strongly convoluted surface of the ciliary body & $100 \mu \mathrm{m}$\\
\hline
$D$ & diffusion coefficient (low molecular species in water) & $\approx 10^{-9} \mathrm{m}/\mathrm{s}$\\
\hline
$T_1$ & characteristic time of diffusion mixing ($T_1 = l^2/D$)& $<2.5 \approx1$\\
\hline
$T_2$ & dwell time of fluid between the folds of ciliary body ($T_2 = V^\star /F_h$) & $\approx 10 \mathrm{min}$\\
\hline
$V_1$ & volume of posterior chamber & $\approx 60 \mathrm{mm}^3$\\
\hline
$T_3$ & dwell time in the posterior chamber ($T_3 = V_1 / F_h$) & $\approx 25 \mathrm{min}$\\
\hline
$L$ & characteristic distance covered by the fluid in posterior chamber & $ \approx 1 \mathrm{cm} $ \\
\hline
$T_4$& diffusion time in the posterior chamber ($T_1 = L^2/D$) & $\approx 2 \times 10^2 \mathrm{min}$
\\
\hline
$Q_e$ & solute outflow from $V^\star$ into the posterior chamber & \\
\hline
$V_2$ & volume of solid structure & \\
\hline
$V_3$ & volume of blood in blood vessels and choroids & \\
\hline
$S$ & area of the posterior chamber-facing secreting surface of the ciliary body & $6 \mathrm{mm}^2 $\\
\hline
$\xi_0$ & epithelial membrane permeability coeff. for low-molecular admixture per unit area ($\xi_0 = \xi/S$) & $ 10^7 \mathrm{
m}/\mathrm{s}$
\\
\hline
$\alpha$ & volume compliance of the eye shell (varies significantly)& $ 1 \mathrm{mm}^3/\mathrm{mmHg}$\\
\hline
$\tau$ & characteristic time of transient processes for hydrodynamic quantities (if $\Delta \Pi_s = \mathrm{Ct}$)&$\approx 1 \mathrm{min}$ (or somewhat higher\\
\hline
\hline
\end{tabular}
\end{center}
Now in the general case where $\Delta \Pi_s$ varies
\begin{center}
\begin{tabular}{|c|p{0.5\textwidth}|l|}
\hline
\hline
$\tau_c$ & characteristic time (stabilization of the concentrations) (if $\Delta \Pi_s $ varies)&an order greater than $\tau$ \\ &&\textit{need checking}\\
\hline
$F_h^\star$ & characteristic influx rate & $\approx 2.4 \mathrm{mm}^3/\mathrm{min}$\\
\hline
$p_t$ & amplitude of $p(\omega)$&\\
\hline
$\Delta \phi$ & phase shift relative to the forcing action& \\
\hline

\end{tabular}
\end{center}
%\section{Asumptions}
%\begin{center}
%\begin{tabular}{|c|p{0.5\textwidth}|}
%\hline
%$R_{\mathrm{inflow}} \ll R_{\mathrm{outflow}}$&\\
%\hline
%$p_{\mathrm{chamber}} = p$&\\
%\hline
%$IOP < P_{\mathrm{chamber}}$&\\
%\hline
%$\Delta_p = \mathrm{Constant}$&osmotic pressure difference accross the membrane\\
%\hline
%$C_{p_2} \approx 2$ &protein concentration in aqueous humor at membrane output \\
%\hline
%$\sigma_p = 1 $ &reflection coefficients for protein\\
%\hline
%$C_1,C_2 < \eps$ & total molar concentration of low molecular components in the blood ($1$) and in the intraocular fluid ($2$) $<$ some  $\eps$\\
%\hline
%$T_1 \ll T_2$ & Diffusion time $\ll$ dwell time \\
%\hline
%$\dd V = \dd U$ & $\dd$ total volume $= \dd$ Volume of aqueous humor in the chambers \\
%\hline
%$\dd V_3 =0 $ & average over rapid oscillation of blood in the choroid\\
%\hline
%\end{tabular}
%\end{center}
%
%
%\section{model}
%\[
%\frac{\dd V}{\dd t} = L_{p} \left( (p_{a} -p) - \sigma_{p} \Delta \Pi_{p} - \sigma_{s} \rho (C_{1}- C_{2}) \right) - \frac{p - p_{e} }{R}
%\]
%\[
%f(v,p) = 0
%\]
%\[
%V^\star \frac{\dd C_2}{\dd t} = - \xi_s (C_2-C_1) + J + \left(1- \sigma_s\right)L_p\left((p_a-p) - \sigma_p \Delta \Pi_p - \sigma_s \rho (C_1-C_2) \right) \frac{C_1 + C_2}{2}
%\]
%Resolution ? (stationary state ?)
%\[
%p\left( L_p + \frac{1}{R} \right) = L_p \left(pa - \sigma_p \Delta \Pi_p - \sigma_s \rho C_1\right) + \frac{p_e}{R} + \sigma_s \rho C_2
%\]
%\[
%p = \alpha_1 + \alpha_2 C_2
%\]
%\[
%\alpha_1 = \frac{1}{L_p + \frac{1}{R}} \left( L_p \left( p_a - \sigma_p \Delta \Pi_p - \sigma_s \rho C_1 \right) + \frac{p_e}{R}\right)
%\]
%\[
%\alpha_2 = \frac{\sigma_s \rho}{L_p + \frac{1}{R}}
%\]
%\[
%\Delta \Pi_s = \rho (C_1 - C2)
%\]
%then
%\[
%C_1 = \frac{\Delta \Pi_s}{\rho} + C_2
%\]
%Resolution : non stationary state ?
%\[
%V = V_0 + \alpha (p - \beta ) - \beta P
%\]
%\[
%\frac{\dd V}{\dd t}= \alpha \frac{\dd p}{\dd t}
%\]
%But we're gonna go with ODE system
%\[
%\alpha \frac{\dd p}{\dd t} = f(p, C_2)
%\]
%with
%\[
%V^\star \frac{\dd C_2}{\dd t} = g(p, C_2)
%\]
%%\section{values}
%%\begin{tabular}{|l|l||l|l|}
%% \hline
%%$ L_p $ &$0.3 mm^ 3$&$p_a$&  $30$-$35$ mmHg (Healthy)\\
%%
%%$\sigma_p$& $ 1$& $\Delta \Pi_p$&$25$ mmHg\\
%%
%%$\sigma_s$& $0.02$-$0.2$& $\sigma_s =  \sigma_s (R)$& $\approx 0.032$\\
%%
%%$p = p(R)$&& $F_h = F_h(R)$&\\
%%
%%$C_1 $& ??& $p_e$& $ 4 $- $8$ mmHg\\
%%
%%?? & $2.5 - 5$ mmHg min / mm$^3$
%%& $\rho$& gaz constant $ \times$ absolute temperature \\
%%
%%$J$ & $0.04$ - $0.18$  $\mu$ mol/min
%%
%%&$\xi$&?\\
%%\hline
%%\end{tabular}\begin{tabular}{|l|l||l|l|}
%% \hline
%%$ L_p $ &$0.3 mm^ 3$&$p_a$&  $30$-$35$ mmHg (Healthy)\\
%%
%%$\sigma_p$& $ 1$& $\Delta \Pi_p$&$25$ mmHg\\
%%
%%$\sigma_s$& $0.02$-$0.2$& $\sigma_s =  \sigma_s (R)$& $\approx 0.032$\\
%%
%%$p = p(R)$&& $F_h = F_h(R)$&\\
%%
%%$C_1 $& ??& $p_e$& $ 4 $- $8$ mmHg\\
%%
%%?? & $2.5 - 5$ mmHg min / mm$^3$
%%& $\rho$& gaz constant $ \times$ absolute temperature \\
%%
%%$J$ & $0.04$ - $0.18$  $\mu$ mol/min
%%
%%&$\xi$&?\\
%%\hline
%%\end{tabular}
%\section{Goal}
%\begin{enumerate}
%\item Paper values $\rightarrow$ p
%\item $R,L_p,\sigma_s$
%\item as 2/ for $p_a = 20 \ldots 40$
%\item $R = R(p)$
%\item $V_3$ choroid
%\item connect with retina
%\end{enumerate}
\bibliographystyle{plain}
\bibliography{bibli}
\end{document}
