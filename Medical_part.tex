
\documentclass[english,12pt]{article}
\usepackage[latin1]{inputenc}
%\usepackage{psfig}
\usepackage[dvips]{graphics}
\usepackage{amsmath,amsthm,amssymb,babel}
\usepackage{color}


\newcommand{\longrightharpoonup}{\,\,-\!\!\!\rightharpoonup}
\def\BBox{\hbox{\vrule height 8pt depth 0pt width 8pt}}
\newtheorem{remark}{Remark}
%\newtheorem{theorem}{Theorem}
\newtheorem{problem}{Problem}
%\newtheorem{lemma}{Lemma}
%\newtheorem{corollary}{Corollary}
\newtheorem{assumption}{Assumption}
\newcommand{\R}{{\if mm {\rm I}\mkern -3mu{\rm R}\else \leavevmode
\hbox{I}\kern -.17em\hbox{R} \fi}}
\newcommand{\blambda}{\mbox{\boldmath{$\lambda$}}}
\newcommand{\bLambda}{\mbox{\boldmath{$\Lambda$}}}
\newcommand{\bmu}{\mbox{\boldmath{$\mu$}}}
\newcommand{\btheta}{\mbox{\boldmath{$\theta$}}}
\newcommand{\bW}{\mbox{\boldmath{$W$}}}
\newcommand{\bM}{\mbox{\boldmath{$M$}}}
\newcommand{\bN}{\mbox{\boldmath{$N$}}}
\newcommand{\bV}{\mbox{\boldmath{$V$}}}
\newcommand{\bP}{\mbox{\boldmath{$P$}}}
\newcommand{\bS}{\mbox{\boldmath{$S$}}}
\newcommand{\bQ}{\mbox{\boldmath{$Q$}}}
\newcommand{\be}{\mbox{\boldmath{$e$}}}
\newcommand{\bu}{\mbox{\boldmath{$u$}}}
\newcommand{\bU}{\mbox{\boldmath{$U$}}}
\newcommand{\bv}{\mbox{\boldmath{$v$}}}
\newcommand{\bw}{\mbox{\boldmath{$w$}}}
\newcommand{\bvs}{\mbox{\boldmath{\footnotesize$v$}}}
\newcommand{\bx}{\mbox{\boldmath{$x$}}}
\newcommand{\bxi}{\mbox{\boldmath{$\xi$}}}
\newcommand{\by}{\mbox{\boldmath{$y$}}}
\newcommand{\bz}{\mbox{\boldmath{$z$}}}
\newcommand{\bI}{\mbox{\boldmath{$I$}}}
\newcommand{\bg}{\mbox{\boldmath{$g$}}}
\newcommand{\fb}{\mbox{\boldmath{$f$}}}
\newcommand{\bh}{\mbox{\boldmath{$h$}}}
\newcommand{\bsigma}{\mbox{\boldmath{$\sigma$}}}
\newcommand{\btau}{\mbox{\boldmath{$\tau$}}}
\newcommand{\bvarphi}{\mbox{\boldmath{$\varphi$}}}
\newcommand{\bvarepsilon}{\mbox{\boldmath{$\varepsilon$}}}
\newcommand{\bnu}{\mbox{\boldmath{$\nu$}}}
\newcommand{\beeta}{\mbox{\boldmath{$\eta$}}}
\newcommand{\bzeta}{\mbox{\boldmath{$\zeta$}}}
\newcommand{\bbeta}{\mbox{\boldmath{$\beta$}}}
\newcommand{\bzero}{\mbox{\boldmath{$0$}}}
\newcommand{\bgamma}{\mbox{\boldmath{$\gamma$}}}
\newcommand{\bpsi}{\mbox{\boldmath{$\psi$}}}
\newcommand{\bphi}{\mbox{\boldmath{$\phi$}}}
\newcommand{\T}{\mathcal{T}}
\newcommand{\cR}{\mbox{{${\cal R}$}}}
\newcommand{\cS}{\mbox{{${\cal S}$}}}
\newcommand{\RR}{\mathbb{R}}
\newcommand{\PP}{\mathcal{P}}
\newcommand{\btu}{\tilde{\mbox{\boldmath{$u$}}}}

\newcommand{\cP}{\mbox{{${\cal P}$}}}

\newtheorem{theorem}{Theorem}[section]
\newtheorem{lemma}[theorem]{Lemma}
\newtheorem{corollary}[theorem]{Corollary}
\newtheorem{definition}[theorem]{Definition}
\newtheorem{proposition}[theorem]{Proposition}





\renewcommand{\baselinestretch}{1.07}
\textwidth 6in \hoffset=-.4in \textheight=9.2in \voffset=-.9in
\parskip   1ex
\parsep    1ex
\itemsep   1ex





\numberwithin{equation}{section}



%\renewcommand{\theequation}{\arabic{equation}}


\vskip 5mm


\title{\bf Mathematical modeling of the eye to understand and predict the intra-ocular pressure and medical treatment for high intra-ocular pressure}
\date{}

\begin{document}
\maketitle



\begin{abstract}
\noindent 
...
\end{abstract}





\bigskip \noindent {\bf Key words\,:}
...


\vskip25mm
\section{Introduction}\label{S1}







\section{Medical aspect}\label{s2}
\subsection{Description of the eye}
\subsubsection{Aqueous humor }
Eyes are the organs of vision; they allows the conversion of light into impulses in neurons. You can find a simple description a the eye below\\
%\begin{figure}
%\include
%\end{figure}
*figure 1*(eye description)\\
Our main interest resides in the ocular flow, more precisely in the aqueous humor. It is a transparent, gelatinous fluid similar to plasma located in the anterior and posterior chambers of the eye. It is produced by the ciliary epithelium and drains into the Schlemm's canal, knowing that these phenomena should be balanced.
This fluid pressure inside the eye which is generated by the production and drainage of aqueous humor, the so-called intra-ocular pression (IOP), range between 10 and 22 mmHg for human with an average value of 16. It also inflates the globe of the eye.
Note that its value can be measured by tonometry by taking into account the thickness of the cornea in order to give sense to this measure (otherwise, the IOP might be underestimated/overestimated), that is to say using a fine jet of air directed toward the cornea, by an ophtalmologist.\\
Problems begin when elevated IOP which is a major risk factor for vision loss.
\subsubsection{The glaucoma}
The glaucoma, also called the silent thief of sight, is known as the second leading cause of blindness worldwide (1 in 40 adults over 40 years old). This disease can be described as a group of ocular disorders with multi-factorial etiology united by a clinically characteristic optic neuropathy accompanied by a vision loss. As a matter of fact, even though the exact origin of glaucoma is not known yet, there are two kinds of diagnostics: a morphological damage (optic disk damage) and a physiological damage (decrease of the visual field). Note that the damages on the optical nerve are irreversible; if the glaucoma is diagnosed too late the patient will become blind whatever is done.\\
*figure 2 (morphological) and 3(physiological)*\\
Until now, the major risk for glaucoma is the elevated IOP but it is neither required nor enough. Indeed, a patient could have a glaucoma with a low IOP while a patient with an elevated IOP will never contracts glaucoma. However, the IOP remains the only parameter we can act on, either by surgery or with medications. 

Even so, 25 \% of IOP-treated patient progress to blindness.

\subsection{The treatment}
\subsubsection{list of medications}
The medications have two effects: either decrease the secretion of aqueous humor or increase the elimination of it. It is also possible to combine several treatments in order to decrease even more the IOP.\\
Here is a list of the medications commonly used:\\
\\
\begin{tabular}{|c|c|}
\hline
Decrease the secretion of aqueous humor & Increase the elimination of aqueous humor\\
\hline
beta-adrenergic receptor antagonists & Prostaglandin analogs \\
Alpha2-adrenergic agonists & Miotic agents \\
alpha agonists &  \\
Carbonic anhydrase inhibitors &  \\

\hline
\end{tabular}

\subsubsection{Different effects on patients}
Note that certain drugs could work better on patients depending on their age, gender, ethnic group, other diseases like diabetes, hypertension ...There are too many unknown factors to take into account, since each patient react differently. That means, for now, the prediction of the IOP of an IOP-treated patient is difficult.

\section{The model and numerical simulations}\label{s3}
$$ \frac{dU}{dt}=F_{h}-F_{e}$$
$$ F_{e}= \frac{p-p_{e}}{R}$$
$$F_{h}= L_p \big[ (p_a-p)-\sigma_{p} \Delta\pi_{p}-\sigma_{s} \Delta\pi_{s}\big]$$
$$\Delta\pi_{s}= \rho(C_1-C_{2}) $$
$$ Q_s=\xi_s(C_1-C_{2})+F_h (1-\sigma_s) \bar{C}+J$$
$$ V^{\ast} \frac{dC_{2}}{dt}= Q_s-Q_e=\xi_s(C_1-C_{2})+F_h (1-\sigma_s) \bar{C}+J-F_h C_2$$
$$ \frac{dV}{dt}= L_p \big[ (p_a-p)-\sigma_{p} \Delta\pi_{p}-\sigma_{s} \Delta\pi_{s}\big]-\frac{p-p_{e}}{R}$$ 
$$  V^{\ast} \frac{dC_{2}}{dt}= \xi_s(C_1-C_{2})+J+\big[ L_p (p_a-p)-\sigma_{p} \Delta\pi_{p}-\sigma_{s} \rho (C_1-C_2)\big]
 \big[(1-\sigma_s)\frac{C_1+C_2}{2}-C_2\big]$$

\begin{thebibliography}{11}

\end{thebibliography}


\end{document}
